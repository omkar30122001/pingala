\documentclass[journal,12pt,twocolumn]{IEEEtran}
%
\usepackage{setspace}
\usepackage{gensymb}
\usepackage{xcolor}
\usepackage{caption}
%\usepackage{subcaption}
%\doublespacing
\singlespacing

%\usepackage{graphicx}
%\usepackage{amssymb}
%\usepackage{relsize}
\usepackage[cmex10]{amsmath}
\usepackage{mathtools}
%\usepackage{amsthm}
%\interdisplaylinepenalty=2500
%\savesymbol{iint}
%\usepackage{txfonts}
%\restoresymbol{TXF}{iint}
%\usepackage{wasysym}
\usepackage{hyperref}
\usepackage{amsthm}
\usepackage{mathrsfs}
\usepackage{txfonts}
\usepackage{stfloats}
\usepackage{cite}
\usepackage{cases}
\usepackage{subfig}
%\usepackage{xtab}
\usepackage{longtable}
\usepackage{multirow}
%\usepackage{algorithm}
%\usepackage{algpseudocode}
%\usepackage{enumerate}
\usepackage{enumitem}
\usepackage{mathtools}
%\usepackage{iithtlc}
%\usepackage[framemethod=tikz]{mdframed}
\usepackage{listings}
\let\vec\mathbf


%\usepackage{stmaryrd}


%\usepackage{wasysym}
%\newcounter{MYtempeqncnt}
\DeclareMathOperator*{\Res}{Res}
%\renewcommand{\baselinestretch}{2}
\renewcommand\thesection{\arabic{section}}
\renewcommand\thesubsection{\thesection.\arabic{subsection}}
\renewcommand\thesubsubsection{\thesubsection.\arabic{subsubsection}}

\renewcommand\thesectiondis{\arabic{section}}
\renewcommand\thesubsectiondis{\thesectiondis.\arabic{subsection}}
\renewcommand\thesubsubsectiondis{\thesubsectiondis.\arabic{subsubsection}}

%\renewcommand{\labelenumi}{\textbf{\theenumi}}
%\renewcommand{\theenumi}{P.\arabic{enumi}}

% correct bad hyphenation here
\hyphenation{op-tical net-works semi-conduc-tor}

\lstset{
language=Python,
frame=single, 
breaklines=true,
columns=fullflexible
}



\begin{document}
%

\theoremstyle{definition}
\newtheorem{theorem}{Theorem}[section]
\newtheorem{problem}{Problem}
\newtheorem{proposition}{Proposition}[section]
\newtheorem{lemma}{Lemma}[section]
\newtheorem{corollary}[theorem]{Corollary}
\newtheorem{example}{Example}[section]
\newtheorem{definition}{Definition}[section]
%\newtheorem{algorithm}{Algorithm}[section]
%\newtheorem{cor}{Corollary}
\newcommand{\BEQA}{\begin{eqnarray}}
\newcommand{\EEQA}{\end{eqnarray}}
\newcommand{\define}{\stackrel{\triangle}{=}}
\newcommand{\myvec}[1]{\ensuremath{\begin{pmatrix}#1\end{pmatrix}}}
\newcommand{\mydet}[1]{\ensuremath{\begin{vmatrix}#1\end{vmatrix}}}

\bibliographystyle{IEEEtran}
%\bibliographystyle{ieeetr}

\providecommand{\nCr}[2]{\,^{#1}C_{#2}} % nCr
\providecommand{\nPr}[2]{\,^{#1}P_{#2}} % nPr
\providecommand{\mbf}{\mathbf}
\providecommand{\pr}[1]{\ensuremath{\Pr\left(#1\right)}}
\providecommand{\qfunc}[1]{\ensuremath{Q\left(#1\right)}}
\providecommand{\sbrak}[1]{\ensuremath{{}\left[#1\right]}}
\providecommand{\lsbrak}[1]{\ensuremath{{}\left[#1\right.}}
\providecommand{\rsbrak}[1]{\ensuremath{{}\left.#1\right]}}
\providecommand{\brak}[1]{\ensuremath{\left(#1\right)}}
\providecommand{\lbrak}[1]{\ensuremath{\left(#1\right.}}
\providecommand{\rbrak}[1]{\ensuremath{\left.#1\right)}}
\providecommand{\cbrak}[1]{\ensuremath{\left\{#1\right\}}}
\providecommand{\lcbrak}[1]{\ensuremath{\left\{#1\right.}}
\providecommand{\rcbrak}[1]{\ensuremath{\left.#1\right\}}}
\theoremstyle{remark}
\newtheorem{rem}{Remark}
\newcommand{\sgn}{\mathop{\mathrm{sgn}}}
\providecommand{\abs}[1]{\left\vert#1\right\vert}
\providecommand{\res}[1]{\Res\displaylimits_{#1}} 
\providecommand{\norm}[1]{\lVert#1\rVert}
\providecommand{\mtx}[1]{\mathbf{#1}}
\providecommand{\mean}[1]{E\left[ #1 \right]}
\providecommand{\fourier}{\overset{\mathcal{F}}{ \rightleftharpoons}}
\providecommand{\ztrans}{\overset{\mathcal{Z}}{ \rightleftharpoons}}

%\providecommand{\hilbert}{\overset{\mathcal{H}}{ \rightleftharpoons}}
\providecommand{\system}{\overset{\mathcal{H}}{ \longleftrightarrow}}
	%\newcommand{\solution}[2]{\textbf{Solution:}{#1}}
\newcommand{\solution}{\noindent \textbf{Solution: }}
\providecommand{\dec}[2]{\ensuremath{\overset{#1}{\underset{#2}{\gtrless}}}}
\numberwithin{equation}{section}
%\numberwithin{equation}{subsection}
%\numberwithin{problem}{subsection}
%\numberwithin{definition}{subsection}
\makeatletter
\@addtoreset{figure}{problem}
\makeatother

\let\StandardTheFigure\thefigure
%\renewcommand{\thefigure}{\theproblem.\arabic{figure}}
\renewcommand{\thefigure}{\theproblem}


%\numberwithin{figure}{subsection}

\def\putbox#1#2#3{\makebox[0in][l]{\makebox[#1][l]{}\raisebox{\baselineskip}[0in][0in]{\raisebox{#2}[0in][0in]{#3}}}}
     \def\rightbox#1{\makebox[0in][r]{#1}}
     \def\centbox#1{\makebox[0in]{#1}}
     \def\topbox#1{\raisebox{-\baselineskip}[0in][0in]{#1}}
     \def\midbox#1{\raisebox{-0.5\baselineskip}[0in][0in]{#1}}

\vspace{3cm}

\title{ 
%\logo{
%}
Pingala Series
%	\logo{Octave for Math Computing }
}
%\title{
%	\logo{Matrix Analysis through Octave}{\begin{center}\includegraphics[scale=.24]{tlc}\end{center}}{}{HAMDSP}
%}


% paper title
% can use linebreaks \\ within to get better formatting as desired
%\title{Matrix Analysis through Octave}
%
%
% author names and IEEE memberships
% note positions of commas and nonbreaking spaces ( ~ ) LaTeX will not break
% a structure at a ~ so this keeps an author's name from being broken across
% two lines.
% use \thanks{} to gain access to the first footnote area
% a separate \thanks must be used for each paragraph as LaTeX2e's \thanks
% was not built to handle multiple paragraphs
%

\author{ Omkar Raijade %<-this  stops a space
% <-this % stops a space
%\thanks{J. Doe and J. Doe are with Anonymous University.}% <-this % stops a space
%\thanks{Manuscript received April 19, 2005; revised January 11, 2007.}}
}
% note the % following the last \IEEEmembership and also \thanks - 
% these prevent an unwanted space from occurring between the last author name
% and the end of the author line. i.e., if you had this:
% 
% \author{....lastname \thanks{...} \thanks{...} }
%                     ^------------^------------^----Do not want these spaces!
%
% a space would be appended to the last name and could cause every name on that
% line to be shifted left slightly. This is one of those "LaTeX things". For
% instance, "\textbf{A} \textbf{B}" will typeset as "A B" not "AB". To get
% "AB" then you have to do: "\textbf{A}\textbf{B}"
% \thanks is no different in this regard, so shield the last } of each \thanks
% that ends a line with a % and do not let a space in before the next \thanks.
% Spaces after \IEEEmembership other than the last one are OK (and needed) as
% you are supposed to have spaces between the names. For what it is worth,
% this is a minor point as most people would not even notice if the said evil
% space somehow managed to creep in.



% The paper headers
%\markboth{Journal of \LaTeX\ Class Files,~Vol.~6, No.~1, January~2007}%
%{Shell \MakeLowercase{\textit{et al.}}: Bare Demo of IEEEtran.cls for Journals}
% The only time the second header will appear is for the odd numbered pages
% after the title page when using the twoside option.
% 
% *** Note that you probably will NOT want to include the author's ***
% *** name in the headers of peer review papers.                   ***
% You can use \ifCLASSOPTIONpeerreview for conditional compilation here if
% you desire.




% If you want to put a publisher's ID mark on the page you can do it like
% this:
%\IEEEpubid{0000--0000/00\$00.00~\copyright~2007 IEEE}
% Remember, if you use this you must call \IEEEpubidadjcol in the second
% column for its text to clear the IEEEpubid mark.



% make the title area
\maketitle

%\newpage

\tableofcontents

%\renewcommand{\thefigure}{\thesection.\theenumi}
%\renewcommand{\thetable}{\thesection.\theenumi}

\renewcommand{\thefigure}{\theenumi}
\renewcommand{\thetable}{\theenumi}

%\renewcommand{\theequation}{\thesection}


\bigskip

\begin{abstract}
This manual provides a simple introduction to Transforms
\end{abstract}
\section{JEE 2019}
Let 
\begin{align}
	a_n &= \frac{\alpha^{n}-\beta^{n}}{\alpha - \beta}, \quad n \ge 1
	\\
	b_n &= a_{n-1} + a_{n+1}, \quad n \ge 2, \quad b_1 =1
	\label{eq:10-orig-diff}
\end{align}
Verify the following using a python code.
\begin{enumerate}[label=\thesection.\arabic*
,ref=\thesection.\theenumi]
\item 
\begin{align}
	\sum_{k=1}^{n}a_k = a_{n+2}-1, \quad n \ge 1
\end{align}
 \item 
\begin{align}
	\sum_{k=1}^{\infty}\frac{a_k}{10^k} =\frac{10}{89}
\end{align}
 \item 
\begin{align}
	b_n =\alpha^n + \beta^n, \quad n \ge 1
\end{align}
 \item 
\begin{align}
	\sum_{k=1}^{\infty}\frac{b_k}{10^k} =\frac{8}{89}
\end{align}
All four problems are verified in the following code:
\begin{lstlisting}
	wget https://github.com/omkar30122001/pingala/blob/main/1.py
\end{lstlisting}
\end{enumerate}
\section{Pingala Series}
\begin{enumerate}[label=\thesection.\arabic*,ref=\thesection.\theenumi]
\item The {\em one sided} $Z$-transform of $x(n)$ is defined as 
%\cite{proakis_dsp}
\begin{align}
	X^{+}(z) = \sum_{n = 0}^{\infty}x(n)z^{-n}, \quad z \in \mathbb{C}
\label{eq:one-Z}
\end{align}
	\item The {\em Pingala} series is generated using the difference equation 
\begin{align}
	x(n+2) = x\brak{n+1} + x\brak{n},  \quad x(0) = x(1) = 1, n \ge 0
	\label{eq:10-pingala}
\end{align}
Generate a stem plot for $x(n)$. \\
\solution
The following code plots $x(n)$:
\begin{lstlisting}
	wget https://github.com/omkar30122001/pingala/blob/main/2.2.py
\end{lstlisting}
\begin{figure}[!htp]
	\includegraphics[width=\columnwidth]{pingala/blob/main/2.2.pdf}
	\caption{Plot of $x(n)$}
	\label{fig:xn}
\end{figure}
\item 		Find $X^{+}(z)$. \\
\solution
Taking one sided Z-Transform on both sides of eq. 2.2
\begin{equation}
	\mathcal{Z}^{+}\sbrak{x(n + 2)} = \mathcal{Z}^{+}\sbrak{x(n + 1)} + \mathcal{Z}^{+}\sbrak{x(n)}
\end{equation}
\begin{equation}
	{z}^{2}X^{+}\brak{z} - z^{2}x(0) - zx(1) = zX^{+}(z) - zx(0) + X^{+}(z)
\end{equation}
\begin{equation}
	\brak{z^{2} - z - 1}X^{+}(z) = z^{2}
\end{equation}
\begin{equation}
	X^{+}(z) = \frac{1}{1 - z^{-1} - z^{-2}}
\end{equation}
\begin{equation}
	\boxed{X^{+}(z) = \frac{1}{\brak{1 - \alpha{z^{-1}}}\brak{1 - \beta{z^{-1}}}} ,\quad \abs{z} > \alpha}
\end{equation}
\item Find $x(n)$. \\
\solution
Expanding $X^{+}(z)$ using partial fractions
\begin{equation}
	X^{+}(z) = \frac{1}{\brak{\alpha - \beta}z^{-1}}\sbrak{\frac{1}{1 - \alpha{z^{-1}}} - \frac{1}{1 - \beta{z^{-1}}}}
\end{equation}
\begin{equation}
	X^{+}(z) = \frac{1}{\alpha - \beta}\sum_{n=0}^{\infty}\brak{{\alpha}^{n} - {\beta}^{n}}z^{-n+1}
\end{equation}
\begin{equation}
	X^{+}(z) = \sum_{n=1}^{\infty}\frac{{\alpha}^{n} - {\beta}^{n}}{\alpha - \beta}z^{-n+1}
\end{equation}
\begin{equation}
	X^{+}(z) = \sum_{k=0}^{\infty}\frac{{\alpha}^{k+1} - {\beta}^{k+1}}{\alpha - \beta}z^{-k}
\end{equation}
Thus
\begin{equation}
	x(n) = \frac{{\alpha}^{n+1} - {\beta}^{n+1}}{\alpha - \beta}u(n)
\end{equation}
\begin{equation}
	\boxed{x(n) = a_{n+1}u(n)}
\end{equation}
\item Sketch 
\begin{align}
	y(n)	 = x\brak{n-1} + x\brak{n+1},  \quad n \ge 0
	\label{eq:10-orig-diff-rev}
\end{align}
\solution
Following code plots $y(n)$
\begin{lstlisting}
	wget https://github.com/omkar30122001/pingala/blob/main/2.5.py
\end{lstlisting}
\begin{figure}[!htp]
	\includegraphics[width=\columnwidth]{blob/main/2.5.pdf}
	\caption{Plot of $y(n)$}
	\label{fig:yn}
\end{figure}
\item Find $Y^{+}(z)$. \\
\solution
Take one sided Z-Transform on both sides of eq. 2.14
\begin{equation}
	\mathcal{Z}^{+}\sbrak{y(n)} = \mathcal{Z}^{+}\sbrak{x(n + 1)} + \mathcal{Z}^{+}\sbrak{x(n - 1)}
\end{equation}
\begin{equation}
	Y^{+}(z) = zX^{+}(z) - zx(0) + z^{-1}X^{+}(z) + zx(-1)
\end{equation}
\begin{equation}
	Y^{+}(z) = \frac{z + z^{-1}}{1 - z^{-1} - z^{-2}} - z
\end{equation}
\begin{equation}
	\boxed{Y^{+}(z) = \frac{1 + 2z^{-1}}{1 - z^{-1} - z^{-2}}, \quad \abs{z} > \alpha}
\end{equation}
\item Find $y(n)$.\\
\solution
From eq. 2.14
\begin{equation}
	y(n) = \sbrak{x(n -1) + x(n + 1)}u(n)
\end{equation}
Using eq. 2.13
\begin{equation}
	y(n) = \sbrak{a_n + a_{n + 2}}u(n)
\end{equation}
Using eq. 1.2
\begin{equation}
	y(n) = b_{n + 1}u(n)
\end{equation}
Using eq. 1.5 (As it is verified in the code)
\begin{equation}
	\boxed{y(n) = \sbrak{{\alpha}^{n + 1} + {\beta}^{n + 1}}u(n)}
\end{equation}
\end{enumerate}
\section{Power of the Z transform}
\begin{enumerate}[label=\thesection.\arabic*,ref=\thesection.\theenumi]
\item Show that 
\begin{align}
	\sum_{k=1}^{n}a_k = 
	\sum_{k=0}^{n-1}x(n) = x(n)*u(n-1)
\end{align}
\solution
\begin{equation}
	\sum_{k=1}^{n}a_k = \sum_{k=0}^{n-1}x(k)
\end{equation}
\begin{equation}
	\sum_{k=1}^{n}a_k = \sum_{k=-\infty}^{n-1}x(k)
\end{equation}
\begin{equation}
	\sum_{k=1}^{n}a_k = \sum_{k=-\infty}^{\infty}x(k)u(n-1-k)
\end{equation}
\begin{equation}
	\sum_{k=1}^{n}a_k = x(n)\ast u(n-1)
\end{equation}
\item Show that 
\begin{align}
a_{n+2}-1, \quad n \ge 1
\end{align}
can be expressed as 
\begin{align}
	\sbrak{x\brak{n+1}-1}u\brak{n}
\end{align}
\solution
From eq. 2.13
\begin{equation}
	a_{n+2} - 1 = \sbrak{x(n + 1) - 1}, \quad n \ge 0
\end{equation}
Using the definition of $u(n)$
\begin{equation}
	\boxed{a_{n + 2} - 1 = \sbrak{x(n + 1) - 1}u(n)}
\end{equation}
 \item Show that 
\begin{align}
	\sum_{k=1}^{\infty}\frac{a_k}{10^k}= 
	\frac{1}{10}\sum_{k=0}^{\infty}\frac{x\brak{k}}{10^k} =\frac{1}{10}X^{+}\brak{{10}}
\end{align}
\solution
\begin{equation}
	\sum_{k=1}^{\infty}\frac{a_k}{10^{k}} = \frac{1}{10}\sum_{k=0}^{\infty}\frac{a_{k+1}}{10^{k}}
\end{equation}
\begin{equation}
	= \frac{1}{10}\sum_{k=0}^{\infty}\frac{x(k)}{10^{k}}
\end{equation}
Using the definition of one sided Z-Transform
\begin{equation}
	\boxed{\sum_{k=1}^{\infty}\frac{a_k}{10^{k}} = \frac{1}{10}X^{+}(10)}
\end{equation}
 \item Show that 
\begin{align}
	\alpha^n + \beta^n, \quad n \ge 1
\end{align}
can be expressed as 
\begin{align}
	w(n) =\brak{\alpha^{n+1} + \beta^{n+1}}u(n)
\end{align}
		and find $W(z)$. \\
\solution
In eq. 3.14, put $n = k + 1$ and from the definition of $u(n)$, we get
\begin{equation}
	\alpha^{n} + \beta^{n} = \brak{{\alpha}^{k + 1} + {\beta}^{k + 1}}u(k)
\end{equation}
Hence 3.14 can be expressed as
\begin{equation}
	w(n) = \brak{\alpha^{n + 1} + \beta^{n + 1}}u(n)
\end{equation}
And $w(n)$ is same as $y(n)$. Hence
\begin{equation}
	\boxed{W(z) = Y(z) = \frac{1 + 2z^{-1}}{1 - z^{-1} - z^{-2}}}
\end{equation}
 \item Show that 
\begin{align}
	\sum_{k=1}^{\infty}\frac{b_k}{10^k} =
	\frac{1}{10}\sum_{k=0}^{\infty}\frac{y\brak{k}}{10^k} =\frac{1}{10}Y^{+}\brak{{10}}
\end{align}
\solution
\begin{equation}
	\sum_{k=1}^{\infty}\frac{b_k}{10^{k}} = \frac{1}{10}\sum_{k=0}^{\infty}\frac{b_{k+1}}{10^{k}}
\end{equation}
\begin{equation}
	= \frac{1}{10}\sum_{k=0}^{\infty}\frac{y(k)}{10^{k}}
\end{equation}
Using the definition of one sided Z-Transform
\begin{equation}
	\boxed{\sum_{k=1}^{\infty}\frac{b_k}{10^{k}} = \frac{1}{10}Y^{+}(10)}
\end{equation}
\item Solve the JEE 2019 problem. \\
\solution
We know that
\begin{equation}
	\sum_{k=1}^{n}a_k = x(n)\ast u(n - 1)
\end{equation}
But
\begin{equation}
	x(n)\ast u(n - 1)\ztrans X(z)z^{-1}U(z)
\end{equation}
\begin{equation}
	= \frac{z^{-1}}{\brak{1 - z^{-1} - z^{-2}}\brak{1 - z^{-1}}}
\end{equation}
\begin{equation}
	= z\sbrak{\frac{1}{1 - z^{-1} - z^{-2}} - \frac{1}{1 - z^{-1}}}
\end{equation}
\begin{equation}
	\ztrans z\sum_{n=0}^{\infty}\sbrak{x(n) - 1}z^{-n}
\end{equation}
\begin{equation}
	= \sum_{n=0}^{\infty}\sbrak{x(n) - 1}z^{-n + 1}
\end{equation}
\begin{equation}
	= \sum_{n=0}^{\infty}\sbrak{x(n+1)-1}z^{-n}
\end{equation}
From eq. 2.13, we get
\begin{equation}
	\sum_{k=1}^{n}a_k = a_{n+2}-1
\end{equation}
We have already established the remaining options in order in the problems 3.3, 2.7 and 3.5 \\
Therefore, options 1, 2 and 3 are correct and option 4 is incorrect.
\end{enumerate}


\end{document}
